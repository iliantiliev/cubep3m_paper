

\documentclass[useAMS,usenatbib]{mn2e}

% The usenatbib command allows the use of Patrick Daly's natbib.sty for
% cross-referencing.
%
% If you wish to typeset the paper in Times font (if you do not have the
% PostScript Type 1 Computer Modern fonts you will need to do this to get
% smoother fonts in a PDF file) then uncomment the next line
% \usepackage{Times}

%%%%% AUTHORS - PLACE YOUR OWN MACROS HERE %%%%%
%  These Macros are taken from the AAS TeX macro package version 4.0.
%  Include this file in your LaTeX source only if you are not using
%  the AAS TeX macro package and need to resolve the macro definitions
%  in the BibTeX entries returned by the ADS abstract service.
%
%  If you plan not to use this file to resolve the journal macros
%  rather than the whole AAS TeX macro package, you should save the
%  file as ``aas_macros.sty'' and then include it in your paper by
%  using a construct such as:
%\documentstyle[11pt,aas_macros]{article}
%
%  For more information on the AASTeX macro package, please see the URL
%http://www.aas.org/publications/aastex.html
%  For more information about ADS abstract server, please see the URL
%http://adswww.harvard.edu/ads_abstracts.html
%

% Abbreviations for journals.  The object here is to provide authors
% with convenient shorthands for the most ``popular'' (often-cited)
% journals; the author can use these markup tags without being concerned
% about the exact form of the journal abbreviation, or its formatting.
% It is up to the keeper of the macros to make sure the macros expand
% to the proper text.  If macro package writers agree to all use the
% same TeX command name, authors only have to remember one thing, and
% the style file will take care of editorial preferences.  This also
% applies when a single journal decides to revamp its abbreviating
% scheme, as happened with the ApJ (Abt 1991).

\let\jnlstyle=\rm
\def\refjnl#1{{\jnlstyle#1}}

\def\aj{\refjnl{AJ}}                   % Astronomical Journal
\def\araa{\refjnl{ARA\&A}}             % Annual Review of Astron and Astrophys
\def\apj{\refjnl{ApJ}}                 % Astrophysical Journal
\def\apjl{\refjnl{ApJ}}                % Astrophysical Journal, Letters
\def\apjs{\refjnl{ApJS}}               % Astrophysical Journal, Supplement
\def\ao{\refjnl{Appl.~Opt.}}           % Applied Optics
\def\apss{\refjnl{Ap\&SS}}             % Astrophysics and Space Science
\def\aap{\refjnl{A\&A}}                % Astronomy and Astrophysics
\def\aapr{\refjnl{A\&A~Rev.}}          % Astronomy and Astrophysics Reviews
\def\aaps{\refjnl{A\&AS}}              % Astronomy and Astrophysics, Supplement
\def\azh{\refjnl{AZh}}                 % Astronomicheskii Zhurnal
\def\baas{\refjnl{BAAS}}               % Bulletin of the AAS
\def\jrasc{\refjnl{JRASC}}             % Journal of the RAS of Canada
\def\memras{\refjnl{MmRAS}}            % Memoirs of the RAS
\def\mnras{\refjnl{MNRAS}}             % Monthly Notices of the RAS
\def\pra{\refjnl{Phys.~Rev.~A}}        % Physical Review A: General Physics
\def\prb{\refjnl{Phys.~Rev.~B}}        % Physical Review B: Solid State
\def\prc{\refjnl{Phys.~Rev.~C}}        % Physical Review C
\def\prd{\refjnl{Phys.~Rev.~D}}        % Physical Review D
\def\pre{\refjnl{Phys.~Rev.~E}}        % Physical Review E
\def\prl{\refjnl{Phys.~Rev.~Lett.}}    % Physical Review Letters
\def\pasp{\refjnl{PASP}}               % Publications of the ASP
\def\pasj{\refjnl{PASJ}}               % Publications of the ASJ
\def\qjras{\refjnl{QJRAS}}             % Quarterly Journal of the RAS
\def\skytel{\refjnl{S\&T}}             % Sky and Telescope
\def\solphys{\refjnl{Sol.~Phys.}}      % Solar Physics
\def\sovast{\refjnl{Soviet~Ast.}}      % Soviet Astronomy
\def\ssr{\refjnl{Space~Sci.~Rev.}}     % Space Science Reviews
\def\zap{\refjnl{ZAp}}                 % Zeitschrift fuer Astrophysik
\def\nat{\refjnl{Nature}}              % Nature
\def\iaucirc{\refjnl{IAU~Circ.}}       % IAU Cirulars
\def\aplett{\refjnl{Astrophys.~Lett.}} % Astrophysics Letters
\def\apspr{\refjnl{Astrophys.~Space~Phys.~Res.}}
                % Astrophysics Space Physics Research
\def\bain{\refjnl{Bull.~Astron.~Inst.~Netherlands}} 
                % Bulletin Astronomical Institute of the Netherlands
\def\fcp{\refjnl{Fund.~Cosmic~Phys.}}  % Fundamental Cosmic Physics
\def\gca{\refjnl{Geochim.~Cosmochim.~Acta}}   % Geochimica Cosmochimica Acta
\def\grl{\refjnl{Geophys.~Res.~Lett.}} % Geophysics Research Letters
\def\jcp{\refjnl{J.~Chem.~Phys.}}      % Journal of Chemical Physics
\def\jgr{\refjnl{J.~Geophys.~Res.}}    % Journal of Geophysics Research
\def\jqsrt{\refjnl{J.~Quant.~Spec.~Radiat.~Transf.}}
                % Journal of Quantitiative Spectroscopy and Radiative Transfer
\def\memsai{\refjnl{Mem.~Soc.~Astron.~Italiana}}
                % Mem. Societa Astronomica Italiana
\def\nphysa{\refjnl{Nucl.~Phys.~A}}   % Nuclear Physics A
\def\physrep{\refjnl{Phys.~Rep.}}   % Physics Reports
\def\physscr{\refjnl{Phys.~Scr}}   % Physica Scripta
\def\planss{\refjnl{Planet.~Space~Sci.}}   % Planetary Space Science
\def\procspie{\refjnl{Proc.~SPIE}}   % Proceedings of the SPIE

\let\astap=\aap
\let\apjlett=\apjl
\let\apjsupp=\apjs
\let\applopt=\ao


% Psfig/TeX 
\def\PsfigVersion{1.9}
% dvips version
%
% All psfig/tex software, documentation, and related files
% in this distribution of psfig/tex are 
% Copyright 1987, 1988, 1991 Trevor J. Darrell
%
% Permission is granted for use and non-profit distribution of psfig/tex 
% providing that this notice is clearly maintained. The right to
% distribute any portion of psfig/tex for profit or as part of any commercial
% product is specifically reserved for the author(s) of that portion.
%
% *** Feel free to make local modifications of psfig as you wish,
% *** but DO NOT post any changed or modified versions of ``psfig''
% *** directly to the net. Send them to me and I'll try to incorporate
% *** them into future versions. If you want to take the psfig code 
% *** and make a new program (subject to the copyright above), distribute it, 
% *** (and maintain it) that's fine, just don't call it psfig.
%
% Bugs and improvements to trevor@media.mit.edu.
%
% Thanks to Greg Hager (GDH) and Ned Batchelder for their contributions
% to the original version of this project.
%
% Modified by J. Daniel Smith on 9 October 1990 to accept the
% %%BoundingBox: comment with or without a space after the colon.  Stole
% file reading code from Tom Rokicki's EPSF.TEX file (see below).
%
% More modifications by J. Daniel Smith on 29 March 1991 to allow the
% the included PostScript figure to be rotated.  The amount of
% rotation is specified by the "angle=" parameter of the \psfig command.
%
% Modified by Robert Russell on June 25, 1991 to allow users to specify
% .ps filenames which don't yet exist, provided they explicitly provide
% boundingbox information via the \psfig command. Note: This will only work
% if the "file=" parameter follows all four "bb???=" parameters in the
% command. This is due to the order in which psfig interprets these params.
%
%  3 Jul 1991	JDS	check if file already read in once
%  4 Sep 1991	JDS	fixed incorrect computation of rotated
%			bounding box
% 25 Sep 1991	GVR	expanded synopsis of \psfig
% 14 Oct 1991	JDS	\fbox code from LaTeX so \psdraft works with TeX
%			changed \typeout to \ps@typeout
% 17 Oct 1991	JDS	added \psscalefirst and \psrotatefirst
%

% From: gvr@cs.brown.edu (George V. Reilly)
%
% \psdraft	draws an outline box, but doesn't include the figure
%		in the DVI file.  Useful for previewing.
%
% \psfull	includes the figure in the DVI file (default).
%
% \psscalefirst width= or height= specifies the size of the figure
% 		before rotation.
% \psrotatefirst (default) width= or height= specifies the size of the
% 		 figure after rotation.  Asymetric figures will
% 		 appear to shrink.
%
% \psfigurepath#1	sets the path to search for the figure
%
% \psfig
% usage: \psfig{file=, figure=, height=, width=,
%			bbllx=, bblly=, bburx=, bbury=,
%			rheight=, rwidth=, clip=, angle=, silent=}
%
%	"file" is the filename.  If no path name is specified and the
%		file is not found in the current directory,
%		it will be looked for in directory \psfigurepath.
%	"figure" is a synonym for "file".
%	By default, the width and height of the figure are taken from
%		the BoundingBox of the figure.
%	If "width" is specified, the figure is scaled so that it has
%		the specified width.  Its height changes proportionately.
%	If "height" is specified, the figure is scaled so that it has
%		the specified height.  Its width changes proportionately.
%	If both "width" and "height" are specified, the figure is scaled
%		anamorphically.
%	"bbllx", "bblly", "bburx", and "bbury" control the PostScript
%		BoundingBox.  If these four values are specified
%               *before* the "file" option, the PSFIG will not try to
%               open the PostScript file.
%	"rheight" and "rwidth" are the reserved height and width
%		of the figure, i.e., how big TeX actually thinks
%		the figure is.  They default to "width" and "height".
%	The "clip" option ensures that no portion of the figure will
%		appear outside its BoundingBox.  "clip=" is a switch and
%		takes no value, but the `=' must be present.
%	The "angle" option specifies the angle of rotation (degrees, ccw).
%	The "silent" option makes \psfig work silently.
%

% check to see if macros already loaded in (maybe some other file says
% "\input psfig") ...
\ifx\undefined\psfig\else\endinput\fi

%
% from a suggestion by eijkhout@csrd.uiuc.edu to allow
% loading as a style file. Changed to avoid problems
% with amstex per suggestion by jbence@math.ucla.edu

\let\LaTeXAtSign=\@
\let\@=\relax
\edef\psfigRestoreAt{\catcode`\@=\number\catcode`@\relax}
%\edef\psfigRestoreAt{\catcode`@=\number\catcode`@\relax}
\catcode`\@=11\relax
\newwrite\@unused
\def\ps@typeout#1{{\let\protect\string\immediate\write\@unused{#1}}}
%\ps@typeout{psfig/tex \PsfigVersion}

%% Here's how you define your figure path.  Should be set up with null
%% default and a user useable definition.

\def\figurepath{./}
\def\psfigurepath#1{\edef\figurepath{#1}}

%
% @psdo control structure -- similar to Latex @for.
% I redefined these with different names so that psfig can
% be used with TeX as well as LaTeX, and so that it will not 
% be vunerable to future changes in LaTeX's internal
% control structure,
%
\def\@nnil{\@nil}
\def\@empty{}
\def\@psdonoop#1\@@#2#3{}
\def\@psdo#1:=#2\do#3{\edef\@psdotmp{#2}\ifx\@psdotmp\@empty \else
    \expandafter\@psdoloop#2,\@nil,\@nil\@@#1{#3}\fi}
\def\@psdoloop#1,#2,#3\@@#4#5{\def#4{#1}\ifx #4\@nnil \else
       #5\def#4{#2}\ifx #4\@nnil \else#5\@ipsdoloop #3\@@#4{#5}\fi\fi}
\def\@ipsdoloop#1,#2\@@#3#4{\def#3{#1}\ifx #3\@nnil 
       \let\@nextwhile=\@psdonoop \else
      #4\relax\let\@nextwhile=\@ipsdoloop\fi\@nextwhile#2\@@#3{#4}}
\def\@tpsdo#1:=#2\do#3{\xdef\@psdotmp{#2}\ifx\@psdotmp\@empty \else
    \@tpsdoloop#2\@nil\@nil\@@#1{#3}\fi}
\def\@tpsdoloop#1#2\@@#3#4{\def#3{#1}\ifx #3\@nnil 
       \let\@nextwhile=\@psdonoop \else
      #4\relax\let\@nextwhile=\@tpsdoloop\fi\@nextwhile#2\@@#3{#4}}
% 
% \fbox is defined in latex.tex; so if \fbox is undefined, assume that
% we are not in LaTeX.
% Perhaps this could be done better???
\ifx\undefined\fbox
% \fbox code from modified slightly from LaTeX
\newdimen\fboxrule
\newdimen\fboxsep
\newdimen\ps@tempdima
\newbox\ps@tempboxa
\fboxsep = 3pt
\fboxrule = .4pt
\long\def\fbox#1{\leavevmode\setbox\ps@tempboxa\hbox{#1}\ps@tempdima\fboxrule
    \advance\ps@tempdima \fboxsep \advance\ps@tempdima \dp\ps@tempboxa
   \hbox{\lower \ps@tempdima\hbox
  {\vbox{\hrule height \fboxrule
          \hbox{\vrule width \fboxrule \hskip\fboxsep
          \vbox{\vskip\fboxsep \box\ps@tempboxa\vskip\fboxsep}\hskip 
                 \fboxsep\vrule width \fboxrule}
                 \hrule height \fboxrule}}}}
\fi
%
%%%%%%%%%%%%%%%%%%%%%%%%%%%%%%%%%%%%%%%%%%%%%%%%%%%%%%%%%%%%%%%%%%%
% file reading stuff from epsf.tex
%   EPSF.TEX macro file:
%   Written by Tomas Rokicki of Radical Eye Software, 29 Mar 1989.
%   Revised by Don Knuth, 3 Jan 1990.
%   Revised by Tomas Rokicki to accept bounding boxes with no
%      space after the colon, 18 Jul 1990.
%   Portions modified/removed for use in PSFIG package by
%      J. Daniel Smith, 9 October 1990.
%
\newread\ps@stream
\newif\ifnot@eof       % continue looking for the bounding box?
\newif\if@noisy        % report what you're making?
\newif\if@atend        % %%BoundingBox: has (at end) specification
\newif\if@psfile       % does this look like a PostScript file?
%
% PostScript files should start with `%!'
%
{\catcode`\%=12\global\gdef\epsf@start{%!}}
\def\epsf@PS{PS}
%
\def\epsf@getbb#1{%
%
%   The first thing we need to do is to open the
%   PostScript file, if possible.
%
\openin\ps@stream=#1
\ifeof\ps@stream\ps@typeout{Error, File #1 not found}\else
%
%   Okay, we got it. Now we'll scan lines until we find one that doesn't
%   start with %. We're looking for the bounding box comment.
%
   {\not@eoftrue \chardef\other=12
    \def\do##1{\catcode`##1=\other}\dospecials \catcode`\ =10
    \loop
       \if@psfile
	  \read\ps@stream to \epsf@fileline
       \else{
	  \obeyspaces
          \read\ps@stream to \epsf@tmp\global\let\epsf@fileline\epsf@tmp}
       \fi
       \ifeof\ps@stream\not@eoffalse\else
%
%   Check the first line for `%!'.  Issue a warning message if its not
%   there, since the file might not be a PostScript file.
%
       \if@psfile\else
       \expandafter\epsf@test\epsf@fileline:. \\%
       \fi
%
%   We check to see if the first character is a % sign;
%   if so, we look further and stop only if the line begins with
%   `%%BoundingBox:' and the `(atend)' specification was not found.
%   That is, the only way to stop is when the end of file is reached,
%   or a `%%BoundingBox: llx lly urx ury' line is found.
%
          \expandafter\epsf@aux\epsf@fileline:. \\%
       \fi
   \ifnot@eof\repeat
   }\closein\ps@stream\fi}%
%
% This tests if the file we are reading looks like a PostScript file.
%
\long\def\epsf@test#1#2#3:#4\\{\def\epsf@testit{#1#2}
			\ifx\epsf@testit\epsf@start\else
\ps@typeout{Warning! File does not start with `\epsf@start'.  It may not be a PostScript file.}
			\fi
			\@psfiletrue} % don't test after 1st line
%
%   We still need to define the tricky \epsf@aux macro. This requires
%   a couple of magic constants for comparison purposes.
%
{\catcode`\%=12\global\let\epsf@percent=%\global\def\epsf@bblit{%BoundingBox}}
%
%
%   So we're ready to check for `%BoundingBox:' and to grab the
%   values if they are found.  We continue searching if `(at end)'
%   was found after the `%BoundingBox:'.
%
\long\def\epsf@aux#1#2:#3\\{\ifx#1\epsf@percent
   \def\epsf@testit{#2}\ifx\epsf@testit\epsf@bblit
	\@atendfalse
        \epsf@atend #3 . \\%
	\if@atend	
	   \if@verbose{
		\ps@typeout{psfig: found `(atend)'; continuing search}
	   }\fi
        \else
        \epsf@grab #3 . . . \\%
        \not@eoffalse
        \global\no@bbfalse
        \fi
   \fi\fi}%
%
%   Here we grab the values and stuff them in the appropriate definitions.
%
\def\epsf@grab #1 #2 #3 #4 #5\\{%
   \global\def\epsf@llx{#1}\ifx\epsf@llx\empty
      \epsf@grab #2 #3 #4 #5 .\\\else
   \global\def\epsf@lly{#2}%
   \global\def\epsf@urx{#3}\global\def\epsf@ury{#4}\fi}%
%
% Determine if the stuff following the %%BoundingBox is `(atend)'
% J. Daniel Smith.  Copied from \epsf@grab above.
%
\def\epsf@atendlit{(atend)} 
\def\epsf@atend #1 #2 #3\\{%
   \def\epsf@tmp{#1}\ifx\epsf@tmp\empty
      \epsf@atend #2 #3 .\\\else
   \ifx\epsf@tmp\epsf@atendlit\@atendtrue\fi\fi}


% End of file reading stuff from epsf.tex
%%%%%%%%%%%%%%%%%%%%%%%%%%%%%%%%%%%%%%%%%%%%%%%%%%%%%%%%%%%%%%%%%%%

%%%%%%%%%%%%%%%%%%%%%%%%%%%%%%%%%%%%%%%%%%%%%%%%%%%%%%%%%%%%%%%%%%%
% trigonometry stuff from "trig.tex"
\chardef\psletter = 11 % won't conflict with \begin{letter} now...
\chardef\other = 12

\newif \ifdebug %%% turn me on to see TeX hard at work ...
\newif\ifc@mpute %%% don't need to compute some values
\c@mputetrue % but assume that we do

\let\then = \relax
\def\r@dian{pt }
\let\r@dians = \r@dian
\let\dimensionless@nit = \r@dian
\let\dimensionless@nits = \dimensionless@nit
\def\internal@nit{sp }
\let\internal@nits = \internal@nit
\newif\ifstillc@nverging
\def \Mess@ge #1{\ifdebug \then \message {#1} \fi}

{ %%% Things that need abnormal catcodes %%%
	\catcode `\@ = \psletter
	\gdef \nodimen {\expandafter \n@dimen \the \dimen}
	\gdef \term #1 #2 #3%
	       {\edef \t@ {\the #1}%%% freeze parameter 1 (count, by value)
		\edef \t@@ {\expandafter \n@dimen \the #2\r@dian}%
				   %%% freeze parameter 2 (dimen, by value)
		\t@rm {\t@} {\t@@} {#3}%
	       }
	\gdef \t@rm #1 #2 #3%
	       {{%
		\count 0 = 0
		\dimen 0 = 1 \dimensionless@nit
		\dimen 2 = #2\relax
		\Mess@ge {Calculating term #1 of \nodimen 2}%
		\loop
		\ifnum	\count 0 < #1
		\then	\advance \count 0 by 1
			\Mess@ge {Iteration \the \count 0 \space}%
			\Multiply \dimen 0 by {\dimen 2}%
			\Mess@ge {After multiplication, term = \nodimen 0}%
			\Divide \dimen 0 by {\count 0}%
			\Mess@ge {After division, term = \nodimen 0}%
		\repeat
		\Mess@ge {Final value for term #1 of 
				\nodimen 2 \space is \nodimen 0}%
		\xdef \Term {#3 = \nodimen 0 \r@dians}%
		\aftergroup \Term
	       }}
	\catcode `\p = \other
	\catcode `\t = \other
	\gdef \n@dimen #1pt{#1} %%% throw away the ``pt''
}

\def \Divide #1by #2{\divide #1 by #2} %%% just a synonym

\def \Multiply #1by #2%%% allows division of a dimen by a dimen
       {{%%% should really freeze parameter 2 (dimen, passed by value)
	\count 0 = #1\relax
	\count 2 = #2\relax
	\count 4 = 65536
	\Mess@ge {Before scaling, count 0 = \the \count 0 \space and
			count 2 = \the \count 2}%
	\ifnum	\count 0 > 32767 %%% do our best to avoid overflow
	\then	\divide \count 0 by 4
		\divide \count 4 by 4
	\else	\ifnum	\count 0 < -32767
		\then	\divide \count 0 by 4
			\divide \count 4 by 4
		\else
		\fi
	\fi
	\ifnum	\count 2 > 32767 %%% while retaining reasonable accuracy
	\then	\divide \count 2 by 4
		\divide \count 4 by 4
	\else	\ifnum	\count 2 < -32767
		\then	\divide \count 2 by 4
			\divide \count 4 by 4
		\else
		\fi
	\fi
	\multiply \count 0 by \count 2
	\divide \count 0 by \count 4
	\xdef \product {#1 = \the \count 0 \internal@nits}%
	\aftergroup \product
       }}

\def\r@duce{\ifdim\dimen0 > 90\r@dian \then   % sin(x+90) = sin(180-x)
		\multiply\dimen0 by -1
		\advance\dimen0 by 180\r@dian
		\r@duce
	    \else \ifdim\dimen0 < -90\r@dian \then  % sin(-x) = sin(360+x)
		\advance\dimen0 by 360\r@dian
		\r@duce
		\fi
	    \fi}

\def\Sine#1%
       {{%
	\dimen 0 = #1 \r@dian
	\r@duce
	\ifdim\dimen0 = -90\r@dian \then
	   \dimen4 = -1\r@dian
	   \c@mputefalse
	\fi
	\ifdim\dimen0 = 90\r@dian \then
	   \dimen4 = 1\r@dian
	   \c@mputefalse
	\fi
	\ifdim\dimen0 = 0\r@dian \then
	   \dimen4 = 0\r@dian
	   \c@mputefalse
	\fi
%
	\ifc@mpute \then
        	% convert degrees to radians
		\divide\dimen0 by 180
		\dimen0=3.141592654\dimen0
%
		\dimen 2 = 3.1415926535897963\r@dian %%% a well-known constant
		\divide\dimen 2 by 2 %%% we only deal with -pi/2 : pi/2
		\Mess@ge {Sin: calculating Sin of \nodimen 0}%
		\count 0 = 1 %%% see power-series expansion for sine
		\dimen 2 = 1 \r@dian %%% ditto
		\dimen 4 = 0 \r@dian %%% ditto
		\loop
			\ifnum	\dimen 2 = 0 %%% then we've done
			\then	\stillc@nvergingfalse 
			\else	\stillc@nvergingtrue
			\fi
			\ifstillc@nverging %%% then calculate next term
			\then	\term {\count 0} {\dimen 0} {\dimen 2}%
				\advance \count 0 by 2
				\count 2 = \count 0
				\divide \count 2 by 2
				\ifodd	\count 2 %%% signs alternate
				\then	\advance \dimen 4 by \dimen 2
				\else	\advance \dimen 4 by -\dimen 2
				\fi
		\repeat
	\fi		
			\xdef \sine {\nodimen 4}%
       }}

% Now the Cosine can be calculated easily by calling \Sine
\def\Cosine#1{\ifx\sine\UnDefined\edef\Savesine{\relax}\else
		             \edef\Savesine{\sine}\fi
	{\dimen0=#1\r@dian\advance\dimen0 by 90\r@dian
	 \Sine{\nodimen 0}
	 \xdef\cosine{\sine}
	 \xdef\sine{\Savesine}}}	      
% end of trig stuff
%%%%%%%%%%%%%%%%%%%%%%%%%%%%%%%%%%%%%%%%%%%%%%%%%%%%%%%%%%%%%%%%%%%%

\def\psdraft{
	\def\@psdraft{0}
	%\ps@typeout{draft level now is \@psdraft \space . }
}
\def\psfull{
	\def\@psdraft{100}
	%\ps@typeout{draft level now is \@psdraft \space . }
}

\psfull

\newif\if@scalefirst
\def\psscalefirst{\@scalefirsttrue}
\def\psrotatefirst{\@scalefirstfalse}
\psrotatefirst

\newif\if@draftbox
\def\psnodraftbox{
	\@draftboxfalse
}
\def\psdraftbox{
	\@draftboxtrue
}
\@draftboxtrue

\newif\if@prologfile
\newif\if@postlogfile
\def\pssilent{
	\@noisyfalse
}
\def\psnoisy{
	\@noisytrue
}
\psnoisy
%%% These are for the option list.
%%% A specification of the form a = b maps to calling \@p@@sa{b}
\newif\if@bbllx
\newif\if@bblly
\newif\if@bburx
\newif\if@bbury
\newif\if@height
\newif\if@width
\newif\if@rheight
\newif\if@rwidth
\newif\if@angle
\newif\if@clip
\newif\if@verbose
\def\@p@@sclip#1{\@cliptrue}


\newif\if@decmpr

%%% GDH 7/26/87 -- changed so that it first looks in the local directory,
%%% then in a specified global directory for the ps file.
%%% RPR 6/25/91 -- changed so that it defaults to user-supplied name if
%%% boundingbox info is specified, assuming graphic will be created by
%%% print time.
%%% TJD 10/19/91 -- added bbfile vs. file distinction, and @decmpr flag

\def\@p@@sfigure#1{\def\@p@sfile{null}\def\@p@sbbfile{null}
	        \openin1=#1.bb
		\ifeof1\closein1
	        	\openin1=\figurepath#1.bb
			\ifeof1\closein1
			        \openin1=#1
				\ifeof1\closein1%
				       \openin1=\figurepath#1
					\ifeof1
					   \ps@typeout{Error, File #1 not found}
						\if@bbllx\if@bblly
				   		\if@bburx\if@bbury
			      				\def\@p@sfile{#1}%
			      				\def\@p@sbbfile{#1}%
							\@decmprfalse
				  	   	\fi\fi\fi\fi
					\else\closein1
				    		\def\@p@sfile{\figurepath#1}%
				    		\def\@p@sbbfile{\figurepath#1}%
						\@decmprfalse
	                       		\fi%
			 	\else\closein1%
					\def\@p@sfile{#1}
					\def\@p@sbbfile{#1}
					\@decmprfalse
			 	\fi
			\else
				\def\@p@sfile{\figurepath#1}
				\def\@p@sbbfile{\figurepath#1.bb}
				\@decmprtrue
			\fi
		\else
			\def\@p@sfile{#1}
			\def\@p@sbbfile{#1.bb}
			\@decmprtrue
		\fi}

\def\@p@@sfile#1{\@p@@sfigure{#1}}

\def\@p@@sbbllx#1{
		%\ps@typeout{bbllx is #1}
		\@bbllxtrue
		\dimen100=#1
		\edef\@p@sbbllx{\number\dimen100}
}
\def\@p@@sbblly#1{
		%\ps@typeout{bblly is #1}
		\@bbllytrue
		\dimen100=#1
		\edef\@p@sbblly{\number\dimen100}
}
\def\@p@@sbburx#1{
		%\ps@typeout{bburx is #1}
		\@bburxtrue
		\dimen100=#1
		\edef\@p@sbburx{\number\dimen100}
}
\def\@p@@sbbury#1{
		%\ps@typeout{bbury is #1}
		\@bburytrue
		\dimen100=#1
		\edef\@p@sbbury{\number\dimen100}
}
\def\@p@@sheight#1{
		\@heighttrue
		\dimen100=#1
   		\edef\@p@sheight{\number\dimen100}
		%\ps@typeout{Height is \@p@sheight}
}
\def\@p@@swidth#1{
		%\ps@typeout{Width is #1}
		\@widthtrue
		\dimen100=#1
		\edef\@p@swidth{\number\dimen100}
}
\def\@p@@srheight#1{
		%\ps@typeout{Reserved height is #1}
		\@rheighttrue
		\dimen100=#1
		\edef\@p@srheight{\number\dimen100}
}
\def\@p@@srwidth#1{
		%\ps@typeout{Reserved width is #1}
		\@rwidthtrue
		\dimen100=#1
		\edef\@p@srwidth{\number\dimen100}
}
\def\@p@@sangle#1{
		%\ps@typeout{Rotation is #1}
		\@angletrue
%		\dimen100=#1
		\edef\@p@sangle{#1} %\number\dimen100}
}
\def\@p@@ssilent#1{ 
		\@verbosefalse
}
\def\@p@@sprolog#1{\@prologfiletrue\def\@prologfileval{#1}}
\def\@p@@spostlog#1{\@postlogfiletrue\def\@postlogfileval{#1}}
\def\@cs@name#1{\csname #1\endcsname}
\def\@setparms#1=#2,{\@cs@name{@p@@s#1}{#2}}
%
% initialize the defaults (size the size of the figure)
%
\def\ps@init@parms{
		\@bbllxfalse \@bbllyfalse
		\@bburxfalse \@bburyfalse
		\@heightfalse \@widthfalse
		\@rheightfalse \@rwidthfalse
		\def\@p@sbbllx{}\def\@p@sbblly{}
		\def\@p@sbburx{}\def\@p@sbbury{}
		\def\@p@sheight{}\def\@p@swidth{}
		\def\@p@srheight{}\def\@p@srwidth{}
		\def\@p@sangle{0}
		\def\@p@sfile{} \def\@p@sbbfile{}
		\def\@p@scost{10}
		\def\@sc{}
		\@prologfilefalse
		\@postlogfilefalse
		\@clipfalse
		\if@noisy
			\@verbosetrue
		\else
			\@verbosefalse
		\fi
}
%
% Go through the options setting things up.
%
\def\parse@ps@parms#1{
	 	\@psdo\@psfiga:=#1\do
		   {\expandafter\@setparms\@psfiga,}}
%
% Compute bb height and width
%
\newif\ifno@bb
\def\bb@missing{
	\if@verbose{
		\ps@typeout{psfig: searching \@p@sbbfile \space  for bounding box}
	}\fi
	\no@bbtrue
	\epsf@getbb{\@p@sbbfile}
        \ifno@bb \else \bb@cull\epsf@llx\epsf@lly\epsf@urx\epsf@ury\fi
}	
\def\bb@cull#1#2#3#4{
	\dimen100=#1 bp\edef\@p@sbbllx{\number\dimen100}
	\dimen100=#2 bp\edef\@p@sbblly{\number\dimen100}
	\dimen100=#3 bp\edef\@p@sbburx{\number\dimen100}
	\dimen100=#4 bp\edef\@p@sbbury{\number\dimen100}
	\no@bbfalse
}
% rotate point (#1,#2) about (0,0).
% The sine and cosine of the angle are already stored in \sine and
% \cosine.  The result is placed in (\p@intvaluex, \p@intvaluey).
\newdimen\p@intvaluex
\newdimen\p@intvaluey
\def\rotate@#1#2{{\dimen0=#1 sp\dimen1=#2 sp
%            	calculate x' = x \cos\theta - y \sin\theta
		  \global\p@intvaluex=\cosine\dimen0
		  \dimen3=\sine\dimen1
		  \global\advance\p@intvaluex by -\dimen3
% 		calculate y' = x \sin\theta + y \cos\theta
		  \global\p@intvaluey=\sine\dimen0
		  \dimen3=\cosine\dimen1
		  \global\advance\p@intvaluey by \dimen3
		  }}
\def\compute@bb{
		\no@bbfalse
		\if@bbllx \else \no@bbtrue \fi
		\if@bblly \else \no@bbtrue \fi
		\if@bburx \else \no@bbtrue \fi
		\if@bbury \else \no@bbtrue \fi
		\ifno@bb \bb@missing \fi
		\ifno@bb \ps@typeout{FATAL ERROR: no bb supplied or found}
			\no-bb-error
		\fi
		%
%\ps@typeout{BB: \@p@sbbllx, \@p@sbblly, \@p@sbburx, \@p@sbbury} 
%
% store height/width of original (unrotated) bounding box
		\count203=\@p@sbburx
		\count204=\@p@sbbury
		\advance\count203 by -\@p@sbbllx
		\advance\count204 by -\@p@sbblly
		\edef\ps@bbw{\number\count203}
		\edef\ps@bbh{\number\count204}
		%\ps@typeout{ psbbh = \ps@bbh, psbbw = \ps@bbw }
		\if@angle 
			\Sine{\@p@sangle}\Cosine{\@p@sangle}
	        	{\dimen100=\maxdimen\xdef\r@p@sbbllx{\number\dimen100}
					    \xdef\r@p@sbblly{\number\dimen100}
			                    \xdef\r@p@sbburx{-\number\dimen100}
					    \xdef\r@p@sbbury{-\number\dimen100}}
%
% Need to rotate all four points and take the X-Y extremes of the new
% points as the new bounding box.
                        \def\minmaxtest{
			   \ifnum\number\p@intvaluex<\r@p@sbbllx
			      \xdef\r@p@sbbllx{\number\p@intvaluex}\fi
			   \ifnum\number\p@intvaluex>\r@p@sbburx
			      \xdef\r@p@sbburx{\number\p@intvaluex}\fi
			   \ifnum\number\p@intvaluey<\r@p@sbblly
			      \xdef\r@p@sbblly{\number\p@intvaluey}\fi
			   \ifnum\number\p@intvaluey>\r@p@sbbury
			      \xdef\r@p@sbbury{\number\p@intvaluey}\fi
			   }
%			lower left
			\rotate@{\@p@sbbllx}{\@p@sbblly}
			\minmaxtest
%			upper left
			\rotate@{\@p@sbbllx}{\@p@sbbury}
			\minmaxtest
%			lower right
			\rotate@{\@p@sbburx}{\@p@sbblly}
			\minmaxtest
%			upper right
			\rotate@{\@p@sbburx}{\@p@sbbury}
			\minmaxtest
			\edef\@p@sbbllx{\r@p@sbbllx}\edef\@p@sbblly{\r@p@sbblly}
			\edef\@p@sbburx{\r@p@sbburx}\edef\@p@sbbury{\r@p@sbbury}
%\ps@typeout{rotated BB: \r@p@sbbllx, \r@p@sbblly, \r@p@sbburx, \r@p@sbbury}
		\fi
		\count203=\@p@sbburx
		\count204=\@p@sbbury
		\advance\count203 by -\@p@sbbllx
		\advance\count204 by -\@p@sbblly
		\edef\@bbw{\number\count203}
		\edef\@bbh{\number\count204}
		%\ps@typeout{ bbh = \@bbh, bbw = \@bbw }
}
%
% \in@hundreds performs #1 * (#2 / #3) correct to the hundreds,
%	then leaves the result in @result
%
\def\in@hundreds#1#2#3{\count240=#2 \count241=#3
		     \count100=\count240	% 100 is first digit #2/#3
		     \divide\count100 by \count241
		     \count101=\count100
		     \multiply\count101 by \count241
		     \advance\count240 by -\count101
		     \multiply\count240 by 10
		     \count101=\count240	%101 is second digit of #2/#3
		     \divide\count101 by \count241
		     \count102=\count101
		     \multiply\count102 by \count241
		     \advance\count240 by -\count102
		     \multiply\count240 by 10
		     \count102=\count240	% 102 is the third digit
		     \divide\count102 by \count241
		     \count200=#1\count205=0
		     \count201=\count200
			\multiply\count201 by \count100
		 	\advance\count205 by \count201
		     \count201=\count200
			\divide\count201 by 10
			\multiply\count201 by \count101
			\advance\count205 by \count201
			%
		     \count201=\count200
			\divide\count201 by 100
			\multiply\count201 by \count102
			\advance\count205 by \count201
			%
		     \edef\@result{\number\count205}
}
\def\compute@wfromh{
		% computing : width = height * (bbw / bbh)
		\in@hundreds{\@p@sheight}{\@bbw}{\@bbh}
		%\ps@typeout{ \@p@sheight * \@bbw / \@bbh, = \@result }
		\edef\@p@swidth{\@result}
		%\ps@typeout{w from h: width is \@p@swidth}
}
\def\compute@hfromw{
		% computing : height = width * (bbh / bbw)
	        \in@hundreds{\@p@swidth}{\@bbh}{\@bbw}
		%\ps@typeout{ \@p@swidth * \@bbh / \@bbw = \@result }
		\edef\@p@sheight{\@result}
		%\ps@typeout{h from w : height is \@p@sheight}
}
\def\compute@handw{
		\if@height 
			\if@width
			\else
				\compute@wfromh
			\fi
		\else 
			\if@width
				\compute@hfromw
			\else
				\edef\@p@sheight{\@bbh}
				\edef\@p@swidth{\@bbw}
			\fi
		\fi
}
\def\compute@resv{
		\if@rheight \else \edef\@p@srheight{\@p@sheight} \fi
		\if@rwidth \else \edef\@p@srwidth{\@p@swidth} \fi
		%\ps@typeout{rheight = \@p@srheight, rwidth = \@p@srwidth}
}
%		
% Compute any missing values
\def\compute@sizes{
	\compute@bb
	\if@scalefirst\if@angle
% at this point the bounding box has been adjsuted correctly for
% rotation.  PSFIG does all of its scaling using \@bbh and \@bbw.  If
% a width= or height= was specified along with \psscalefirst, then the
% width=/height= value needs to be adjusted to match the new (rotated)
% bounding box size (specifed in \@bbw and \@bbh).
%    \ps@bbw       width=
%    -------  =  ---------- 
%    \@bbw       new width=
% so `new width=' = (width= * \@bbw) / \ps@bbw; where \ps@bbw is the
% width of the original (unrotated) bounding box.
	\if@width
	   \in@hundreds{\@p@swidth}{\@bbw}{\ps@bbw}
	   \edef\@p@swidth{\@result}
	\fi
	\if@height
	   \in@hundreds{\@p@sheight}{\@bbh}{\ps@bbh}
	   \edef\@p@sheight{\@result}
	\fi
	\fi\fi
	\compute@handw
	\compute@resv}

%
% \psfig
% usage : \psfig{file=, height=, width=, bbllx=, bblly=, bburx=, bbury=,
%			rheight=, rwidth=, clip=}
%
% "clip=" is a switch and takes no value, but the `=' must be present.
\def\psfig#1{\vbox {
	% do a zero width hard space so that a single
	% \psfig in a centering enviornment will behave nicely
	%{\setbox0=\hbox{\ }\ \hskip-\wd0}
	%
	\ps@init@parms
	\parse@ps@parms{#1}
	\compute@sizes
	%
	\ifnum\@p@scost<\@psdraft{
		%
		\special{ps::[begin] 	\@p@swidth \space \@p@sheight \space
				\@p@sbbllx \space \@p@sbblly \space
				\@p@sbburx \space \@p@sbbury \space
				startTexFig \space }
		\if@angle
			\special {ps:: \@p@sangle \space rotate \space} 
		\fi
		\if@clip{
			\if@verbose{
				\ps@typeout{(clip)}
			}\fi
			\special{ps:: doclip \space }
		}\fi
		\if@prologfile
		    \special{ps: plotfile \@prologfileval \space } \fi
		\if@decmpr{
			\if@verbose{
				\ps@typeout{psfig: including \@p@sfile.Z \space }
			}\fi
			\special{ps: plotfile "`zcat \@p@sfile.Z" \space }
		}\else{
			\if@verbose{
				\ps@typeout{psfig: including \@p@sfile \space }
			}\fi
			\special{ps: plotfile \@p@sfile \space }
		}\fi
		\if@postlogfile
		    \special{ps: plotfile \@postlogfileval \space } \fi
		\special{ps::[end] endTexFig \space }
		% Create the vbox to reserve the space for the figure.
		\vbox to \@p@srheight sp{
		% 1/92 TJD Changed from "true sp" to "sp" for magnification.
			\hbox to \@p@srwidth sp{
				\hss
			}
		\vss
		}
	}\else{
		% draft figure, just reserve the space and print the
		% path name.
		\if@draftbox{		
			% Verbose draft: print file name in box
			\hbox{\frame{\vbox to \@p@srheight sp{
			\vss
			\hbox to \@p@srwidth sp{ \hss \@p@sfile \hss }
			\vss
			}}}
		}\else{
			% Non-verbose draft
			\vbox to \@p@srheight sp{
			\vss
			\hbox to \@p@srwidth sp{\hss}
			\vss
			}
		}\fi	



	}\fi
}}
\psfigRestoreAt
\let\@=\LaTeXAtSign




\usepackage{graphicx}
\usepackage{epsfig}
\usepackage{amssymb}
\usepackage{amsmath}
\usepackage{amsfonts}
\usepackage{txfonts}

\title[{\small} CUBEP3M: High Performance P$^{3}$M N-body code]{{\small} CUBEP3M: High Performance P$^{3}$M N-body code}
\author[Joachim Harnois-D\'{e}raps, Ue-Li Pen, Hugh Merz, Ilian T. Iliev]{Joachim Harnois-D\'{e}raps$^{1,2}$ 
\thanks{E-mail: jharno@cita.utoronto.ca},  Ue-Li Pen$^{1}$ \thanks{E-mail: pen@cita.utoronto.ca}, 
Hugh Merz$^{3}$ \thanks{E-mail: merz@sharcnet.ca} and  Ilian T. Iliev$^{4}$ \thanks{E-mail: i.t.iliev@sussex.ac.uk}\\
%\footnotemark[1]\thanks{This file has been amended to
%highlight the proper use of \LaTeXe\ code with the class file.
%These changes are for illustrative purposes and do not reflect the
%original paper by A. V. Raveendran.}\\
$^{1}$Canadian Institute for Theoretical Astrophysics, University of
Toronto, M5S 3H8, Canada\\
$^{2}$Department of Physics, University of Toronto, M5S 1A7, Ontario,  Canada\\
$^{3}$Department of XXX\\
$^{4}$Department of XXX\\
$^{5}$Department of XXX}

\begin{document}

%\date{Accepted 1988 December 15. Received 1988 December 14; in original form 1988 October 11}
\date{\today}

\pagerange{\pageref{firstpage}--\pageref{lastpage}} \pubyear{2011}

\maketitle

\label{firstpage}

\begin{abstract}
This paper presents {\small CUBEP3M}, an upgraded version of {\small PMFAST}, 
a two-mesh gravity solver that was already among the fastest N-body codes. 
Among the principal changes, the Poisson solver includes particle-particles interactions
at the sub-grid level and across neighbouring cells, plus each level of the volume decomposition is now cubical.
Force kernel have been improved for enhanced matching of the two mesh contributions near the cutoff length.
We discuss the structure of the code, its accuracy, and its scaling performance.
In addition, many utilities have been added, including a runtime halofinder,
particle identification tags and non-Gaussian initial conditions generator, and we briefly describe their implementation strategy
and accuracy, when applicable . 

\end{abstract}

\begin{keywords}
N-body simulations --- Large scale structure of Universe --- Dark matter
\end{keywords}

\section{Introduction}

Many physical and astrophysical systems are subject to non-linear dynamics
and rely on N-body simulations to describe the evolution of bodies. 
One of the main field of application is the modelling of large scale structures, 
which are driven by the sole force of gravity. Recent observations of the cosmic microwave background
\citep{WMAP5-7} of galaxy clustering \citep{2df, SDSS, WiggleZ, BOSS} of weak gravitational lensing \citep{CFHTLS, SDSS}
and of supernovae redshift-distance relations all point towards a standard model of cosmology, in which dark energy
and collisionless dark matter occupy more than 95 per cent of the total energy density of the universe.
In such a paradigm, pure N-body code are perfectly suited to describe the dynamics, as long as we
understands where and how the baryonic fluid feeds back on the dark matter structure.
The next generation of measurements aim at constraining the cosmological parameters at the per cent level, 
and the theoretical understanding of the non-linear dynamics that govern structure formation heavily relies 
on numerical simulations. 

For instance, a measurement of the baryonic acoustic oscillation (BAO) dilation scale can provide
tight constraints on the dark energy equation of scale \citep{Seo2003,2005}.
The most optimal estimates of the uncertainty requires the knowledge of the 
matter power spectrum covariance matrix, which is only accurate when measured from a large sample 
of N-body simulations \citep{RH2005, japan2,japan3, Ngan}.
For the same reasons, the most accurate estimates of weak gravitational lensing signal is obtained
by propagating photons in past light cones that are extracted from simulated density fields \citep{HarnoisLudo, Takahashi }.
{\bf (Ilian, could you say something about reionization here?)}

The basic problem that is addressed with N-body codes is a time evolution of an ensemble of $N$ particles
that is subject to gravitational attraction. The brute force calculation requires $O(N^{2})$ operation, a cost that 
exceeds the memory and speed of current machines for large problems.
Solving the problem  on a mesh \citep{Hockney} reduces to $O(N\mbox{log}N)$ the number of operations,
as it is possible to solve for the particle-mesh (PM) interaction with fast Fourier transforms techniques, 
typically using high performance libraries such as {\small FFTW} \citep{FFTW}.


With the advent of large computing facilities, parallel coding has now become common practice, 
and N-body codes have evolved both in performance and complexity. 
Many codes have opted for a tree algorithm \citep{Gadjet, Gadjet2, TPM, GOTPM}, in which 
the local resolution increases with the density of the matter field. 
These have the advantage to balance the load across the computing units, which enable the calculation of high density regions. 
The drawback is a significant loss in speed, which can be only partly recovered by turning off the tree algorithm. 
Unlike Hydra \citep{couchman1991} and the PM code by \cite{FerrelBertschinger1995},
 these are not designed to perform large scale PM calculations. 

{\small PMFAST} (\cite{PMFAST}, MPT hereafter) is one of the first code that is designed such as to optimize the PM algorithm,
both in terms of speed and memory usage. It uses a two-level mesh algorithm based on the gravity solver of \cite{TracPen2003},
The long range gravitational force is computed on a  grid four times coarser, such as to minimize the communication time
and to fit in system's memory. The short range is computed locally on a finer mesh, which can be as tight as the memory allows.
This enable the code to evolve very large cosmological systems both rapidly and accurately, on relatively modest clusters.




{\bf (Describe briefly   Hydra, PM code by japanese, and others...)}










\section{Review of the Code Structure}
\label{sec:structure}


An optimal large scale N-body code must address many challenges: minimize the memory footprint to allow larger dynamical range,
minimize the passing of information across computing nodes, reduce and accelerate the memory accesses to the large scale arrays, 
make efficient use of high performance libraries to speed up standard calculations like Fourier transforms, just to name a few.
In the realm of parallel programming, high efficiency  can be assessed when a high load is balanced across all processors
most of the time. In this section, we present the general strategies adopted to address these challenges\footnote{ 
Many originate directly from MPT and were preserved in {\small CUBEP3M};
those will be briefly mentioned, and we shall refer the reader to the original PMFAST paper for greater details.}.
We start with a walkthrough the code flow, and briefly discuss some specific sections that depart from standard N-body codes,
while referring the reader to future sections for detailed discussions on selected topics.


As mentioned in the Introduction section, {\small CUBEP3M} is a {\small FORTRAN90} 
N-body code that solves Poisson's equation on a two-level mesh, 
with sub-cell accuracy thanks to particle-particle interactions. 
The code has extensions that departs from this basic scheme, and
we shall come back to these later, but for the moment, we adopt the 
standard configuration. 
The long range component of the gravity force is solved on the coarse grid, 
and is global in the sense that the calculations require knowledge about the full simulated volume.
The short range force and the particle-particle interactions are computed in parallel on local volumes\footnote{To make this possible, the fine grid arrays are constructed such as to support parallel reading and writing. In practice, this is done by adding an additional dimension to the relevant arrays, such that each {\small CPU} accesses a unique memory location.} or {\it tiles},
and for maximal efficiency, the number of tiles per node should be at least equal to the number of available processors.
Given the available freedom in the parallel configuration, it is generally good practice to maximize the number of {\small OPENMP} threads and minimize the number of {\small MPI} processes, as the information exchange between cores that are part of the same motherboard is generally much faster.
As mentioned in MPT, the computation of the short range force requires each tile to store the fine grid density in a buffer surface around the physical volume it is assigned. The thickness of this surface must be larger than the cutoff length, and we find that a 24 cells deep buffer
is a good compromise between memory usage and accuracy {\bf (Hugh, is that correct?)}
 

%\subsection{Memory foot-print and communication strategy}
%\label{subsec:memory}

Because the coarse grid arrays require $4^3$ times less memory per node, 
the bulk of the foot-print is concentrated in a handful of fine grid arrays.
In addition, some of these are required for intermediate steps of the calculations only, 
hence it is possible to hide some of the coarse grid arrays\footnote{ This memory recycling is done with `equivalence' statements in {\small F90}}.   
We present here the largest arrays used by the code:
\begin{enumerate}
\item{{\tt xv} stores the position and velocity of each particle} 
\item{{\tt ll} stores the linked-list that accelerate the access to particles in a local domain}
\item{{\tt rho\_f} and {\tt cmplx\_rho\_f} store 
the local fine grid density  in real and Fourier space respectively}
\item{{\tt force\_f} stores the force of gravity (short range only) along the three Cartesian directions}
\item{{\tt kern\_f} stores the fine grid force kernel in the three directions}
\end{enumerate}

%\subsection{Code overview}
%\label{subsec:overview}

The code flow is presented in Fig. \ref{fig:structure} and \ref{fig:particle_mesh}.
Before entering the main loop, the code starts with an initialization stage, 
in which many declared variables are assigned default values,
the redshift checkpoints are read, the {\small FFTW} plans are created, and the {\small MPI} communicators are defined.
The phase-space array  is obtained from the output of the initial conditions generator,
and the force kernels on both grids are constructed from the specific geometry of the simulation.
For clarity, all these operations are collected under the subroutine call {\tt initialize} in Fig. \ref{fig:structure}, 
although they are actually distinct calls in the code.

Each iteration of the main loop starts with the {\tt timestep} subroutine, 
which proceeds to a determination of the redshift jump by comparing the step size constraints from each
force components and from the scale factor.
The cosmic expansion is found by Taylor expanding Friedmann's equation up to the third order in the scale factor,
and can accommodate constant or running equation of state of dark energy.
The force of gravity is then solved  in the {\tt particle\_mesh} subroutine,
which first updates the positions and velocities of the dark matter particles, exchange with neighbouring nodes those that have exited to volume,
creates a new linked list, then solve Poisson's equation.  This subroutine is conceptually identical to that of {\small PMFAST}, 
with the exception  that {\small CUBEP3M} decomposes the volume into cubes (as opposed to slabs). 
The loop over tiles and the particle exchange are thus performed in three dimensions.
{\bf (Should I mention leapfrog here or anywhere else?)}
When the short range and pp forces have been calculated on all tiles, the code exits the parallel {\small OPENMP} loop
and proceeds to the long range. This section of the code is also parallelized in many occasions, but, unfortunately, the current {\small MPI-FFTW}
do not allow multi-threading. There is thus an inevitable loss of efficiency during each global Fourier transforms, during which
only the head processor is active. Other libraries such as {\small P3DFFT} or  Intel {\small MKL-$\alpha$} currently permit this extra level of parallelization,
and it is our plan to migrate to one of these in the near future.

\begin{figure}
\begin{verbatim}
program cubep3m
   call initialize
   do
       call timestep
       call particle_mesh
       if(checkpoint_step) then
          call checkpoint
       elseif(last_step)
          exit
       endif
   enddo
   call finalize
end program cubep3m
\end{verbatim}
\caption{Overall structure of the code.}
\label{fig:structure}
\end{figure}

\begin{figure}
\begin{verbatim}
subroutine particle_mesh
   call update_position
   call link_list
   call particle_pass
   !$omp parallel do
   do tile = 1, tiles_node
      call rho_f_ngp
      call cmplx_rho_f
      call kernel_multiply_f
      call force_f
      call update_velocity_f
      if(pp = .true.) then       
         call link_list_pp
         call force_pp
         call update_velocity_pp
         if(extended_pp = .true.) then
            call link_list_pp_extended
            call force_pp_extended
            call update_velocity_pp_extended       
         endif
      endif
   end do
   !$omp end parallel do
   call rho_c_ngp
   call cmplx_rho_c
   call kernel_multiply_c
   call force_c
   call update_velocity_c      
   delete_buffers
end subroutine particle_mesh
\end{verbatim}
\caption{Overall structure of the two-level mesh algorithm. We have included the section that concerns the extended pp force calculation to show that it follows the same basic logic. We mention here that the three linked list arrays are distinct entities, and their structure differ slightly.  }
\label{fig:particle_mesh}
\end{figure}


If the current redshift corresponds to one of the checkpoints, the code advances all particles to their final location
and writes them to file. Similarly, the code can compute two-dimensional projections of the density field, halo catalogues (see section \ref{sec:halo} for details), and can compute the power spectrum on the coarse grid at run time {\bf (is this correct?)}. 
The code exits the loop when it has reached the final redshift, it then wraps up the {\small FFTW} plans 
and clears the {\small MPI} communicators. We have collected those operations under the subroutine {\tt finalize} for concision.


{\bf (Anything else to say in this section? Units by \citep{2004NewA....9..443T}?)}

\section{Poisson Solver}
\label{sec:Poisson}


This section reviews how Poisson's equation is solved on a double-mesh configuration. 
Many parts of the algorithm are identical to {\small PMFAST}, hence we refer the reader 
to section 2 of MPT for more details. In cubep3m, the mass default assignment scheme are
a `cloud-in-cell' (cic) interpolation for the coarse grid,  and a `nearest-grid-point' interpolation 
for the fine grid \citep{1981csup.book.....H}. Particle-particle interactions are not interpolated, but a sharp cutoff kills the force
for pairs closer to a tenth of a grid cell. 

The force of gravity on a mesh can be computed either with a gravitational potential  kernel $\omega_{\phi} ({\bf x})$
  or a force  kernel $\omega_{F} ({\bf x})$.
Gravity fields are curl-free, which allows us to relate the potential $\phi({\bf x})$ to the source term via Poisson's equation: 
\begin{eqnarray}
\nabla^{2}\phi({\bf x}) = 4 \pi G \rho({\bf x})
\label{eq:poisson}
\end{eqnarray}
$G$ being Newton's constant. We solve this equation in Fourier space, where we write
\begin{eqnarray}
 \tilde{\phi}({\bf k}) = \frac{4 \pi G \tilde{\rho}({\bf k})}{- k^{2}} \equiv \tilde{\omega}_{\phi}({\bf k})\tilde{\rho}({\bf k})
\label{eq:poissonFourier}
\end{eqnarray}
The potential in real space is then obtained with an inverse Fourier transform, and the kernel becomes $\omega_{\phi} ({\bf x}) = -G/r$.
Using the convolution theorem, we can write
\begin{eqnarray}
 \phi({\bf x}) = \int \rho({\bf x'}) \omega_{\phi}({\bf x'} - {\bf x}) d{\bf x'}
\label{eq:poisson_solution_pot}
\end{eqnarray}
Although this approach is fast, it involves a finite differentiation which enhances the numerical noise.
We therefore opt for a force kernel, which is more accurate but has the inconvenient to require four extra Fourier transforms.
In this case, we must solve the convolution in three dimensions:
\begin{eqnarray}
 F({\bf x}) = - m {\bf \nabla} \phi({\bf x})   = \int \rho({\bf x'}) {\bf  \omega}_{F}({\bf x'} - {\bf x}) d{\bf x'}                                      
\label{eq:poisson_solution_force}
\end{eqnarray}
The differentiation does not affect the density since it only acts on un-prime variables,
and the force kernel is given by 
\begin{eqnarray}
{\bf  \omega}_{F}({\bf x}) \equiv - {\bf \nabla}\omega_{\phi}({\bf x}) = - \frac{mG \hat{\bf r}}{r^{2}}
\end{eqnarray}

Following the spherically symmetric matching technique of MPT (section 2.1), 
we split  the force kernel into two components, for the short and long range respectively, and 
match the overlapping region with a polynomial. Namely, we have:
\begin{eqnarray}
{\bf \omega}_{s}({\bf r}) = \begin{cases} {\bf \omega}_{F}(r) -  {\bf \beta}(r) &\mbox{if  } \mbox{$r$ $\le$ $r_{c}$ } \\
0 & \mbox{otherwise} 
\end{cases}
\end{eqnarray}
and
\begin{eqnarray}
{\bf \omega}_{l}(r) = \begin{cases} {\bf \beta}(r) &\mbox{if  } \mbox{$r$ $\le$ $r_{c}$ } \\
 {\bf\omega}_{F}(r)  &\mbox{otherwise} 
\end{cases}
\end{eqnarray}
The vector $ {\bf \beta}(r)$ is related to the fourth order polynomial that is used in the potential case by
 $ {\bf \beta} = - {\bf \nabla} \alpha(r)$. The coefficients are found by matching the boundary conditions at $r_{c}$ up to the second derivative,
 and we get
  \begin{eqnarray}
   {\bf \beta}(r) = \bigg[ -\frac{7 r}{4 r_{c}^{3}} + \frac{3 r^{3}}{4 r^{5}}\bigg] \hat{\bf r}
  \end{eqnarray}

Because these calculations are performed on two grids of different resolution, a sampling window function must be convoluted 
both with the density and the kernel (see [Eq. 7-8] of MPT).
When matching the two  force kernels, it was realized that close to the cutoff region, the long range force is always on the low side, whereas 
the short range force is scattered across the theoretical $1/r^2$ value. These behaviours are purely features of the CIC and NGP interpolation scheme 
respectively. We identified a small range surrounding the cutoff length, in which we empirically adjust both kernels such as to improve 
the match. Namely, for $14 \le r \le 16$, ${\bf \omega}_{s}({\bf r}) \rightarrow {\bf \omega}_{s}({\bf r})*0.985$,
and for  $12 \le r \le 16$, ${\bf \omega}_{l}({\bf r}) \rightarrow {\bf \omega}_{l}({\bf r})*1.2$.
The two fudge factors were found by performing force measurements on two particles randomly placed in the volume.
 
In addition, the long range force is subject to a correction that fixes {\bf (what exactly fixes the LRCKCORR flag?)}.

{\bf (Anything else here?)}

\section{Scaling Performances}

Real time versus size, bottle neck, description of the largest runs, etc.
{\bf (Ilian, I will need your input here as well. Hugh, do you have anything you would like to share 
concerning the largest runs you did on Blue Gene? )} 


\section{Systematics and Accuracy}
\label{sec:systematics}

To discuss : 

1) random shift to shuffle systematic clustering at the grid scale, 

2) choice of initial redshift, which can lead to large truncation error if too early, but poor inaccurate Zel'dovich if too late 
{\bf JD, here you could discuss the ra\_max tempering...} 

3) limits of resolution caused by softening length,

4) Poisson shot noise if too few particles or unrelaxed systems, 5) finite box size effects (with beat coupling?)
 
Show plots of pairwise force (transverse and radial), density force, power spectrum, 
mention the work on covariance matrix by Ngan, Harnoisetal.




\section{Runtime Halo Finder}
\label{sec:halo}

Spherical overdensities, search algorithm, 
provided halo information, halo bias, comparison to PS and ST.
{\bf (Ilian, You can lead the way here...)}
\section{Beyond the standard configuration}
\label{sec:extensions}

The preceding descriptions and discussions apply to the standard configuration of the code, 
as described at the beginning of section \ref{sec:structure}. A few extensions have been recently developed
in order to enlarge the range of applications of {\small CUBEP3M}, and this section briefly 
describe the most important improvements.

\subsection{Initial Conditions}
\label{subsec:init}

As mention in section \ref{sec:structure}, the code start off by reading a set of initial conditions.
These correspond to a  $6 \times N$ phase-space array, where $N$ is the number of particles in the
local node. Although most applications so far were based on initial conditions that follow Gaussian statistics,
we have developed a non-Gaussian initial condition generator that we briefly describe in this section. 
{\bf Vincent, here you go!} 
\section{Other tools}

Particle ID, initial condition generator, projections, power spectrum code,  various expansions

\subsection{Extended range of the pp force}
\label{subsec:extendedpp}

One of the main source of error in the calculation of the force occurs when on the smallest scales of the fine grid.
The approximation by which particles in a neighbouring mesh grid can be placed at the centre of the cell
is less accurate, which cause a maximal scatter around the exact $1/r^2$ law.
A solution to minimize this error consists in extending the pp force calculation outside a single cell,
which inevitably reintroduces a $N^2$ number of operations. Our goal is to add the flexibility to have a code
that runs slower, but produces results with a higher precision. 

To allow this feature, we  have to choose how far outside a cell we want the exact pp force.  
Since the force kernels on both meshes are organized in terms of grids, the simplest way to implement this 
feature is to shut down the mesh kernels in a region of specified size, and allow the pp force to extend therein.
Concretely, these regions are constructed as cubic layers of fine mesh grids around a central cell; 
the freedom we have is to choose the number of such layers.
 
 To speed up the access to all particles within the domain of computation, we construct a thread safe linked list
 to be constructed and accessed in parallel by each core of the system, but this time with a head-of-chain that points to the first particle in the current fine mesh cell. We then loop over all fine grids, accessing the particles contained therein and inside each fine grid cells for which we killed the mesh kernels,
 we compute the separation and the force between each pairs and update their velocities simultaneously with Newton's third law. 
 To avoid double counting, we loop only over the fine mesh neighbours that produce non-redundant contributions. Namely, for a central cell located at 
 $(x_1, y_1, z_1)$, we only consider the neighbours $(x_2, y_2, z_2)$ that satisfy the following conditions:
 \begin{itemize}
 \item{$z_2 \ge z_1$ always}
 \item{if $z_2 = z_1$, then $y_2 \ge y_1$, otherwise we also allow $y_2 < y_1$} 
 \item{if $z_2 = z_1$ and $y_2 = y_1$, then we enforce $x_2 > x_1$}
 \end{itemize}
 The case where all three coordinates are equal is already calculated in the standard configuration of the code.
 
 To quantify the accuracy improvement versus computing time requirements, we performed the following test.
 We generate a set of initial conditions at a starting redshift of $z = XXX$, with a box size equal to $ XXX h^{-1}\mbox{Mpc}$,
 and with $XXX^{3}$ particles. We evolve the particles to $z=0$ with different ranges for the pp calculation, and compare 
 the resulting power spectra. For the results to be meaningful, we also need to use the same random seed for the random number generator,
 such that the only difference between different runs is the range of the pp force.
{\bf (JD, insert here your P(k) plot and a description of what you observe, in terms of accuracy improvement and real time cpu usage...)}
\subsection{Generalization of the dark energy equation of state}
\label{subsec:runningomegal} 

\section{Conclusion}

\section{Acknowledgements}

%\input{Appendix}
\bibliographystyle{mn2e}
\bibliography{mybib3}{}
%\bibliographystyle{amsplain}

\bsp

\label{lastpage}


\end{document}
